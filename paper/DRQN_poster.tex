
\documentclass[final]{beamer}

\usepackage[scale=1.24]{beamerposter} % Use the beamerposter package for laying out the poster

\usetheme{confposter} % Use the confposter theme supplied with this template

\setbeamercolor{block title}{fg=ngreen,bg=white} % Colors of the block titles
\setbeamercolor{block body}{fg=black,bg=white} % Colors of the body of blocks
\setbeamercolor{block alerted title}{fg=white,bg=dblue!70} % Colors of the highlighted block titles
\setbeamercolor{block alerted body}{fg=black,bg=dblue!10} % Colors of the body of highlighted blocks
% Many more colors are available for use in beamerthemeconfposter.sty

%-----------------------------------------------------------
% Define the column widths and overall poster size
% To set effective sepwid, onecolwid and twocolwid values, first choose how many columns you want and how much separation you want between columns
% In this template, the separation width chosen is 0.024 of the paper width and a 4-column layout
% onecolwid should therefore be (1-(# of columns+1)*sepwid)/# of columns e.g. (1-(4+1)*0.024)/4 = 0.22
% Set twocolwid to be (2*onecolwid)+sepwid = 0.464
% Set threecolwid to be (3*onecolwid)+2*sepwid = 0.708

\newlength{\sepwid}
\newlength{\onecolwid}
\newlength{\twocolwid}
\newlength{\threecolwid}
\setlength{\paperwidth}{48in} % A0 width: 46.8in
\setlength{\paperheight}{36in} % A0 height: 33.1in
\setlength{\sepwid}{0.024\paperwidth} % Separation width (white space) between columns
\setlength{\onecolwid}{0.22\paperwidth} % Width of one column
\setlength{\twocolwid}{0.464\paperwidth} % Width of two columns
\setlength{\threecolwid}{0.708\paperwidth} % Width of three columns
\setlength{\topmargin}{-0.5in} % Reduce the top margin size
%-----------------------------------------------------------

\usepackage{graphicx}  % Required for including images

\usepackage{booktabs} % Top and bottom rules for tables

%----------------------------------------------------------------------------------------
%	TITLE SECTION 
%----------------------------------------------------------------------------------------

\title{Deep Q-Learning With Recurrent Neural Networks} % Poster title

\author{Clare Chen, Vincent Ying, Dillon Laird} % Author(s)

\institute{Stanford, 2016} % Institution(s)

%----------------------------------------------------------------------------------------

\begin{document}

\addtobeamertemplate{block end}{}{\vspace*{2ex}} % White space under blocks
\addtobeamertemplate{block alerted end}{}{\vspace*{2ex}} % White space under highlighted (alert) blocks

\setlength{\belowcaptionskip}{2ex} % White space under figures
\setlength\belowdisplayshortskip{2ex} % White space under equations

\begin{frame}[t] % The whole poster is enclosed in one beamer frame

\begin{columns}[t] % The whole poster consists of three major columns, the second of which is split into two columns twice - the [t] option aligns each column's content to the top

\begin{column}{\sepwid}\end{column} % Empty spacer column

\begin{column}{\onecolwid} % The first column

%----------------------------------------------------------------------------------------
%	OBJECTIVES
%----------------------------------------------------------------------------------------

\begin{alertblock}{Abstract}
Deep reinforcement learning models have proven to be successful at learning
control policies image inputs. They have, however, struggled with learning
policies that require longer term information. Recurrent neural network
architectures have be used in tasks dealing with longer term dependencies
between data points. We investigate these architectures to overcome the
difficulties arising from learning policies with long term dependencies.
\end{alertblock}

%----------------------------------------------------------------------------------------
%	QUICK REVISION
%----------------------------------------------------------------------------------------

\begin{block}{Introduction}
    \begin{itemize}
        \item Recent advances in reinforcement Learning have led to human-level
              or greater performance on a wide variety of games (e.g. Atari 2600
              Games). 
        \item Deep Q-networks are limited in that they learn a mapping from a single
              previous state which constist of a small number of game screens.
        \item We explore the concept of a deep recurrent Q-network (DRQN), a
              combination of a recurrent neural network (RNN) and a deep
              Q-network (DQN) 
        \item In addition to vanilla RNN architectures we also examine augmented RNN
              architectures such as attention RNNs. 
    \end{itemize}
\end{block}

%------------------------------------------------

\begin{figure}[h]
    \begin{minipage}{0.8\textwidth}
        \centering
        \includegraphics[scale=0.5]{Qbert}
        \centering
        \includegraphics[scale=0.5]{MsPacman}
        \centering
        \includegraphics[scale=0.5]{MontezumaRevenge}
    \end{minipage}
    \caption{Q*bert, Ms. Pac-Man and Montezuma's Revenge}
\end{figure}

%\begin{figure}
%\includegraphics[width=0.8\linewidth]{1.jpg}
%\caption{Graph of $f(x)=ax^2|_{\{0.1, 0.3, 1.0, 3.0\}}$}
%\end{figure}

%----------------------------------------------------------------------------------------

\end{column} % End of the first column

\begin{column}{\sepwid}\end{column} % Empty spacer column

\begin{column}{\twocolwid} % Begin a column which is two columns wide (column 2)

\begin{columns}[t,totalwidth=\twocolwid] % Split up the two columns wide column

\begin{column}{\onecolwid}\vspace{-.6in} % The first column within column 2 (column 2.1)

%----------------------------------------------------------------------------------------
%	MATERIALS
%----------------------------------------------------------------------------------------

\begin{block}{Background}

%\begin{enumerate}
%\item Rearrange the equation into the standard $ax^2+bx+c$ form.
%\item Write down two brackets: $(x\ \ \ )(x\ \ \ )$
%\item Find two numbers that multiply to give 'c' and add or subtract to give 'b' (ignoring signs).
%\item Put the numbers in brackets and choose their signs.
%\end{enumerate}

\end{block}

%----------------------------------------------------------------------------------------

\end{column} % End of column 2.1

\begin{column}{\onecolwid}\vspace{-.6in} % The second column within column 2 (column 2.2)

%----------------------------------------------------------------------------------------
%	P
%----------------------------------------------------------------------------------------

\begin{block}{Attention Deep Recurrent Q-Learning}
\end{block}
\end{column} % End of column 2.2


\end{columns} % End of the split of column 2 - any content after this will now take up 2 columns width

%\begin{figure}
%    \includegraphics[width=0.4\linewidth]{DRQN_attn}
%    \caption{Architecture of the Attention DRQN}
%\end{figure}

\begin{alertblock}{Box}
Text or image?
\end{alertblock} 

%----------------------------------------------------------------------------------------

\begin{columns}[t,totalwidth=\twocolwid] % Split up the two columns wide column again

\begin{column}{\onecolwid} % The first column within column 2 (column 2.1)

\begin{block}{Deep Recurrent Q-Learning}
\begin{itemize}
    \item The architecture of DRQN augments DQN's fully connected layer with a LSTM.
    \item We accomplish this by looking at the last $L$ states:
        $$ \{s_{t-(L-1)}, \dots, s_{t}\} $$
    \item We feed these into a convultion neural network (CNN) to get intermediate
        outputs and finally send those through the RNN:
        $$\text{CNN}(s_{t-i}) = x_{t-i}$$
        $$\text{RNN}(x_{t-i}, h_{t-i-1}) = h_{t-i}$$
    \item The final output is used to predict the $Q$ value.
\end{itemize}

\end{block}

%----------------------------------------------------------------------------------------

\end{column} % End of column 2.1

\begin{column}{\onecolwid} % The second column within column 2 (column 2.2)

\begin{block}{Title}
\end{block}

%----------------------------------------------------------------------------------------

\end{column} % End of column 2.2

\end{columns} % End of the split of column 2

\end{column} % End of the second column

\begin{column}{\sepwid}\end{column} % Empty spacer column

\begin{column}{\onecolwid} % The third column

%----------------------------------------------------------------------------------------
%	CONCLUSION
%----------------------------------------------------------------------------------------

\begin{block}{Title}
\end{block}


%----------------------------------------------------------------------------------------
%	ACKNOWLEDGEMENTS
%----------------------------------------------------------------------------------------

\setbeamercolor{block title}{fg=red,bg=white} % Change the block title color

\begin{block}{References}

% table

%\begin{table}
%\vspace{2ex}
%\begin{tabular}{l l l l}
%\toprule
%\textbf{verb} & \textbf{noun} & \textbf{meaning}\\
%\midrule
%add & addition & $+$ \\
%subtract & subtraction & $-$ \\
%multiply & multiplication & $\cdot$ \\
%divide & division & $\div$ \\
%solve & solution & getting answer \\
%substitute & substitution & $t=x^2$ \\
%\bottomrule
%\end{tabular}
%\caption{Word Formation}
%\end{table}


\end{block}

% orange box
%\setbeamercolor{block alerted title}{fg=black,bg=norange} % Change the alert block title colors
%\setbeamercolor{block alerted body}{fg=black,bg=white} % Change the alert block body colors

\begin{alertblock}{Box}

\begin{itemize}
    \item item
\end{itemize}

\end{alertblock}


%----------------------------------------------------------------------------------------

\end{column} % End of the third column

\end{columns} % End of all the columns in the poster

\end{frame} % End of the enclosing frame

\end{document}
